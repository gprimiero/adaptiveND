\section{Formula-Condition pairs in ND}
In this chapter I propose a possible simplification of the
adaptive natural deduction system proposed in the last note.

For the sake of simplicity I will demonstrate the approach on
the basis of minimal logic. It should generalize for any LLL
though. 

Prawitz proposes the following inference rules for minimal
logic:\footnote{Some are later reformulated in terms of
  deduction rules so to give a precise account of the
  discharging of assumptions. This could be done analogously
  here, but for the sake of simplicity I stick to the simple
  representation for sketching this idea.}
\begin{equation*}
  \begin{array}[h]{c@{~~~~~~~}c}
    \infer[\& I]{A\wedge B}{A & B} & %
    \infer[\& E]{A}{A & B} ~~~\infer[\& E]{B}{A & B} \\[.2cm]
    \infer[\vee I]{A \vee B}{A} ~~~\infer[\vee I]{A \vee B}{B}
    & %
    \infer[\vee E]{C}{A\vee B & \deduce{C}{(A)} &
      \deduce{C}{(B)}} \\[.2cm]
    \infer[\supset I]{A \supset B}{\deduce{B}{(A)}} & %
    \infer[\supset E]{B}{A & A \supset B}   
  \end{array}
\end{equation*}

In the adaptive treatment we need to represent conditions in
which the (defeasible) assumptions are stored with the help
of which formulas are derived. Hence, instead of only listing
formulas in the antecedent spots and the conclusion spot of
rules, we list formula-condition pairs $A.\Delta$ where
$\Delta$ is a finite set of abnormalities.

We now translate the rules of the given LLL by means of the
following scheme:
\begin{equation*}
  \vcenter{\infer{B}{A_1 & \ldots & A_n}} ~~\leadsto~~
  \vcenter{\infer{B.\bigcup_{i=1}^n \Delta_i}{A_1.\Delta_1 & \ldots &
      A_n.\Delta_n}}
\end{equation*}
In the specific case of minimal logic we end up with:
\begin{equation*}
  \begin{array}[h]{c@{~~~~~~~}c}
    \infer[\& IU]{A \wedge B.\Delta_A\cup\Delta_B}{A.\Delta_A
      & B.\Delta_B} &  %
    \infer[\& EU]{A.\Delta_A\cup\Delta_B}{A.\Delta_A & B.\Delta_B} ~~~\infer[\& EU]{B.\Delta_A\cup\Delta_B}{A.\Delta_A & B.\Delta_B} \\[.2cm]
    \infer[\vee IU]{A \vee B.\Delta}{A.\Delta} ~~~\infer[\vee IU]{A \vee B.\Delta}{B.\Delta}
    & %
    \infer[\vee EU]{C.\Delta\cup\Delta_A\cup\Delta_B}{A\vee B.\Delta & \deduce{C.\Delta_A}{(A)} &
      \deduce{C.\Delta_B}{(B)}} \\[.2cm]
    \infer[\supset IU]{A \supset B.\Delta}{\deduce{B.\Delta}{(A)}} & %
    \infer[\supset EU]{B.\Delta\cup\Delta'}{A.\Delta & A \supset B.\Delta'}   
  \end{array}
\end{equation*}
We add the following conditional rule:
\begin{equation*}
  \infer[RC]{A.\Delta\cup\Delta'}{A \vee
    \Dab(\Delta).\Delta'} 
\end{equation*}

Let in the following $\tau$ be a derivation tree constructed
by the given rules.

In the derivation trees we allow only for top-nodes of the
form $A.\emptyset$. 

In order to define the marking of nodes in derivation trees
we need some more terminology. 

Let $\Dab(\Delta_1).\emptyset, \Dab(\Delta_2).\emptyset,
\ldots$ be all nodes of the form $\Dab(\Delta).\emptyset$ in
$\tau$. We say $\Dab(\Delta)$ is a minimal \Dab-formula in
$\tau$ iff there is no $\Delta_i \subset \Delta$ and there is
a $\Delta_j = \Delta$. Let $U(\tau) = \Delta_1 \cup \Delta_2
\cup \ldots$.

\begin{defn}[Marking of nodes in $\tau$, Reliability]
  A node $A.\Delta$ is marked in $\tau$ iff $\Delta \cap
  U(\tau) \neq \emptyset$.
\end{defn}

Let's take a look at our example from the previous section:
\begin{equation*}
  \vcenter{\infer{A.\{!B\}}{A\vee {!B}.\emptyset}} ~~\leadsto~~ %
  \vcenter{\infer{\checkmark~ A\wedge (!B \vee
      {!C}).\{!B\}}{\infer{\checkmark~ A.\{!B\}}{A\vee{!B}.\emptyset} & !B
      \vee {!C}.\emptyset}} ~~\leadsto~~ %
  \vcenter{\infer{A \wedge {!C}.\{!B\}}{\infer{A\wedge (!B \vee
        {!C}).\{!B\}}{\infer{A.\{!B\}}{A\vee{!B}.\emptyset} & !B
        \vee {!C}.\emptyset} & !C.\emptyset}}
\end{equation*}

Again, we're interested in a stable notion of derivability:

Let $A_1.\emptyset, A_2.\emptyset,\ldots$ be the charged
top-nodes of $\tau$. We say $\tau$ is an $\Gamma$-tree iff
$\{A_1, A_2,\ldots\} \subseteq \Gamma$. Similarly, we call an
extension $\tau'$ of $\tau$ a $\Gamma$-extension iff $\tau'$
is a $\Gamma$-tree.

\begin{defn}[Final derivability in $\tau$]
  \emph{$A$ is finally derived in $\tau$ from $\Gamma$} iff
  (i) $\tau$ is a $\Gamma$-tree, (ii) there is an unmarked
  node $n$ with $A.\Delta$, (iii) for every
  $\Gamma$-extension $\tau'$ of $\tau$ in which $n$ is marked
  there is a further $\Gamma$-extension $\tau''$ such that
  $n$ is unmarked.
\end{defn}

\begin{defn}
  $\Gamma \vd{AdND} A$ iff $A$ is finally derived from
  $\Gamma$ in some tree $\tau$.
\end{defn}


%%% Local Variables: 
%%% mode: latex
%%% TeX-master: "giuseppe-ND"
%%% End: 
